\documentclass[english,man]{apa6}

\usepackage{amssymb,amsmath}
\usepackage{ifxetex,ifluatex}
\usepackage{fixltx2e} % provides \textsubscript
\ifnum 0\ifxetex 1\fi\ifluatex 1\fi=0 % if pdftex
  \usepackage[T1]{fontenc}
  \usepackage[utf8]{inputenc}
\else % if luatex or xelatex
  \ifxetex
    \usepackage{mathspec}
    \usepackage{xltxtra,xunicode}
  \else
    \usepackage{fontspec}
  \fi
  \defaultfontfeatures{Mapping=tex-text,Scale=MatchLowercase}
  \newcommand{\euro}{€}
\fi
% use upquote if available, for straight quotes in verbatim environments
\IfFileExists{upquote.sty}{\usepackage{upquote}}{}
% use microtype if available
\IfFileExists{microtype.sty}{\usepackage{microtype}}{}

% Table formatting
\usepackage{longtable, booktabs}
\usepackage{lscape}
% \usepackage[counterclockwise]{rotating}   % Landscape page setup for large tables
\usepackage{multirow}		% Table styling
\usepackage{tabularx}		% Control Column width
\usepackage[flushleft]{threeparttable}	% Allows for three part tables with a specified notes section
\usepackage{threeparttablex}            % Lets threeparttable work with longtable

% Create new environments so endfloat can handle them
% \newenvironment{ltable}
%   {\begin{landscape}\begin{center}\begin{threeparttable}}
%   {\end{threeparttable}\end{center}\end{landscape}}

\newenvironment{lltable}
  {\begin{landscape}\begin{center}\begin{ThreePartTable}}
  {\end{ThreePartTable}\end{center}\end{landscape}}

  \usepackage{ifthen} % Only add declarations when endfloat package is loaded
  \ifthenelse{\equal{\string man}{\string man}}{%
   \DeclareDelayedFloatFlavor{ThreePartTable}{table} % Make endfloat play with longtable
   % \DeclareDelayedFloatFlavor{ltable}{table} % Make endfloat play with lscape
   \DeclareDelayedFloatFlavor{lltable}{table} % Make endfloat play with lscape & longtable
  }{}%



% The following enables adjusting longtable caption width to table width
% Solution found at http://golatex.de/longtable-mit-caption-so-breit-wie-die-tabelle-t15767.html
\makeatletter
\newcommand\LastLTentrywidth{1em}
\newlength\longtablewidth
\setlength{\longtablewidth}{1in}
\newcommand\getlongtablewidth{%
 \begingroup
  \ifcsname LT@\roman{LT@tables}\endcsname
  \global\longtablewidth=0pt
  \renewcommand\LT@entry[2]{\global\advance\longtablewidth by ##2\relax\gdef\LastLTentrywidth{##2}}%
  \@nameuse{LT@\roman{LT@tables}}%
  \fi
\endgroup}


\ifxetex
  \usepackage[setpagesize=false, % page size defined by xetex
              unicode=false, % unicode breaks when used with xetex
              xetex]{hyperref}
\else
  \usepackage[unicode=true]{hyperref}
\fi
\hypersetup{breaklinks=true,
            pdfauthor={},
            pdftitle={Supplement to Common or Distinct Attention Mechanisms for Contrast and Assimilation?},
            colorlinks=true,
            citecolor=blue,
            urlcolor=blue,
            linkcolor=black,
            pdfborder={0 0 0}}
\urlstyle{same}  % don't use monospace font for urls

\setlength{\parindent}{0pt}
%\setlength{\parskip}{0pt plus 0pt minus 0pt}

\setlength{\emergencystretch}{3em}  % prevent overfull lines

\ifxetex
  \usepackage{polyglossia}
  \setmainlanguage{}
\else
  \usepackage[english]{babel}
\fi

% Manuscript styling
\captionsetup{font=singlespacing,justification=justified}
\usepackage{csquotes}
\usepackage{upgreek}



\usepackage{tikz} % Variable definition to generate author note

% fix for \tightlist problem in pandoc 1.14
\providecommand{\tightlist}{%
  \setlength{\itemsep}{0pt}\setlength{\parskip}{0pt}}

% Essential manuscript parts
  \title{Supplement to `Common or Distinct Attention Mechanisms for Contrast and
Assimilation?'}

  \shorttitle{Contrast and Assimilation Supplement}


  \author{Hope K. Snyder\textsuperscript{1}, Sean M. Rafferty\textsuperscript{1}, Julia M. Haaf\textsuperscript{1}, \& Jeffery N. Rouder\textsuperscript{2,1}}

  \def\affdep{{"", "", "", ""}}%
  \def\affcity{{"", "", "", ""}}%

  \affiliation{
    \vspace{0.5cm}
          \textsuperscript{1} University of Missouri\\
          \textsuperscript{2} University of California, Irvine  }

  \authornote{
    \newcounter{author}
    This document was written in R-Markdown with code for data analysis
    integrated into the text. The Markdown script is open and freely
    available at
    \url{https://github.com/PerceptionAndCognitionLab/ctx-flanker/tree/public/papers/current}.
    The data were \emph{born open} (Rouder, 2016) and are freely available
    at
    \url{https://github.com/PerceptionCognitionLab/data1/tree/master/ctxIndDif/flankerMorph4}

                      Correspondence concerning this article should be addressed to Hope K. Snyder, 205 McAlester Hall, University of Missouri, Columbia, MO 65211. E-mail: \href{mailto:hks7w2@mail.missouri.edu}{\nolinkurl{hks7w2@mail.missouri.edu}}
                                              }


  \keywords{Inhibition, Selective Attention, Contrast Effects, Assimilation Effects \\

    
  }




  \usepackage{enumerate}
  \usepackage{amsmath, amssymb,amsthm,amsfonts}
  \usepackage{multicol}
  \usepackage[usenames,dvipsnames]{pstricks}
  \usepackage{mathtools}

\begin{document}

\maketitle

\setcounter{secnumdepth}{0}



This document is the supplement to \enquote{Common or Distinct Attention
Mechanisms for Contrast and Assimilation?}. It provides the
specification and analysis of a hierarchical Bayesian probit model for
assessing the correlation across individuals' abilities to inhibit
distractors in assimilation and contrast contexts.

\section{Model Specification}\label{model-specification}

Let \(Y=0,1\) denote whether a response is \enquote{A} or \enquote{H},
respectively. Further, let \(Y_{ijk\ell m}\) denote the response for the
\(i\)th participant, \(i=1,\ldots,I\), in the \(j\)th context type
(\(j=1,2\) for word and letter contexts, respectively), for the \(k\)th
context direction (\(k=1,2\), for contexts that promote \enquote{A} and
\enquote{H} responses, respectively), for the \(\ell\)th target,
\(\ell=1,\ldots,L\), and for the \(m\)th replicate,
\(m=1,\ldots ,M_{ijk\ell}\). The context assignments are displayed in
the following table:

\begin{longtable}[c]{@{}clcc@{}}
\toprule
\begin{minipage}[b]{0.12\columnwidth}\centering\strut
~
\strut\end{minipage} &
\begin{minipage}[b]{0.09\columnwidth}\raggedright\strut
\strut\end{minipage} &
\begin{minipage}[b]{0.15\columnwidth}\centering\strut
k=1
\strut\end{minipage} &
\begin{minipage}[b]{0.15\columnwidth}\centering\strut
k=2
\strut\end{minipage}\tabularnewline
\midrule
\endhead
\begin{minipage}[t]{0.12\columnwidth}\centering\strut
\strut\end{minipage} &
\begin{minipage}[t]{0.09\columnwidth}\raggedright\strut
\strut\end{minipage} &
\begin{minipage}[t]{0.15\columnwidth}\centering\strut
A-promoting
\strut\end{minipage} &
\begin{minipage}[t]{0.15\columnwidth}\centering\strut
H-promoting
\strut\end{minipage}\tabularnewline
\begin{minipage}[t]{0.12\columnwidth}\centering\strut
\textbf{j=1}
\strut\end{minipage} &
\begin{minipage}[t]{0.09\columnwidth}\raggedright\strut
word
\strut\end{minipage} &
\begin{minipage}[t]{0.15\columnwidth}\centering\strut
C\_T
\strut\end{minipage} &
\begin{minipage}[t]{0.15\columnwidth}\centering\strut
T\_E
\strut\end{minipage}\tabularnewline
\begin{minipage}[t]{0.12\columnwidth}\centering\strut
\textbf{j=2}
\strut\end{minipage} &
\begin{minipage}[t]{0.09\columnwidth}\raggedright\strut
letter
\strut\end{minipage} &
\begin{minipage}[t]{0.15\columnwidth}\centering\strut
H
\strut\end{minipage} &
\begin{minipage}[t]{0.15\columnwidth}\centering\strut
A
\strut\end{minipage}\tabularnewline
\bottomrule
\end{longtable}

Observations \(Y_{ijk\ell m}\) are dichotomous. To model the effect of
covariates in dichotomous observations, we use a probit-regression
specification:
\[Y_{ijk\ell m} \stackrel{ind}{\sim} \text{Bernoulli}\left[\Phi(\mu_{ijk\ell})\right].\]

Here, \(\Phi\), the cumulative distribution function of the standard
normal, is the link, and \(\mu_{ijk\ell} \in (\infty,\infty)\) is the
combined effect of people, conditions, and the target.

To model individual inhibition effects, we additively decompose
\(\mu_{ijk\ell}\) into the effect of the target for a particular
participant, \(\gamma_{i\ell}\), the individual's \emph{assimilation}
effect when the background context is a word, \(\alpha_i\) and the
individual's \emph{contrast} effect when the background context is a
letter frame, \(\beta_i\). The decomposition is:
\[\mu_{ijk\ell}=\gamma_{i\ell}+v_jx_k\alpha_i+(1-v_j)x_k\beta_i.\]

The quantities \(v_j\) and \(x_k\) are indicators of the context type
and direction, as follows: \[ v_j=
  \begin{dcases}
      0 & \quad \text{if} \quad j =2  \text{ (letters)}\\
      1 & \quad \text{if} \quad j =1 \text{ (word)}
  \end{dcases}
  \qquad
  x_k=
  \begin{dcases}
      \frac{-1}{2} & \quad \text{if} \quad k =1  \text{ (A-promoting)}\\
      \frac{1}{2}  & \quad \text{if} \quad k =2 \text{ (H-promoting)}
  \end{dcases}
\]

Prior distributions are needed for
\(\alpha_i, \beta_i, \text{ and } \gamma_{i\ell}\). We start with
\(\gamma_{i\ell}\), the effect of the target for a particular
participant. These parameters are not of primary concern, and we use a
broad hierarchical prior given by:
\(\gamma_{i\ell} \sim \mbox{N}(\nu_\ell,\delta_\ell)\) with hyper priors
of \(\nu_{\ell} \sim \mbox{N}(0,1)\) and
\(\delta_{\ell} \sim \mbox{Inverse Gamma}(.5,.01)\).\\
The effects of interest are an individual's assimilation effect,
\(\alpha_i\), and an individual's contrast effect, \(\beta_i\). We
follow the development in Rouder et al. (2007) who describe Bayesian
analyses in these settings. The joint prior over these parameter is
\[\begin{pmatrix}
  \alpha_i\\
  \beta_i
\end{pmatrix} = \mbox{N}(\boldsymbol{\lambda},\boldsymbol{\Sigma}),\]
where \(\boldsymbol{\lambda}\) is the vector
\((\mu_{\alpha},\mu_{\beta})\) and
\(\boldsymbol{\Sigma}=\begin{pmatrix} \Sigma_{1,1} & \Sigma_{1,2} \\ \Sigma_{2,1} & \Sigma_{2,2} \end{pmatrix}\)
is a covariance matrix.

Following Rouder et al. (2007), the hyperpriors for \(\mu_{\alpha}\) and
\(\mu_{\beta}\) are given by \[\mu_{\alpha} \sim N(0,1),\]
\[ \mu_{\beta} \sim N(0,1). \] The hyperprior for
\(\boldsymbol{\Sigma}\) is
\[\boldsymbol{\Sigma} \sim \mbox{Inverse Wishert}(3,\Omega),\] where
\(\Omega = \begin{vmatrix}.05 & 0 \\ 0 & .05 \end{vmatrix}\). A key
property of the above specification is that population covariance in
\(\boldsymbol{\Sigma}\) is an explicit parameter given by
\(\Sigma_{1,2}\). Hence, a population correlation is well defined, and
subsequent inferences generalize to new data from new participants.

The choices we make above are all reasonable given the expected degree
of variability in proportions. The standard normal prior in the probit
space on \(\mu_{\beta}\) and \(\mu_{\alpha}\) corresponds to flat priors
in the space of proportions, and the shape value of \(3\) in the Inverse
Wishert corresponds to a flat prior on the population-level correlation
coefficient. The only substantive choice are the values \(.05\) in the
scale of the Inverse Wishert. This small value of precision corresponds
to a vaguely informative prior on effects. Additional details are
provided in Rouder et al. (2007).

\section{Analyses and Results}\label{analyses-and-results}

The estimation of probit-transformed parameters follows from Albert and
Chib (1995). Accordingly, a new set of latent variables is introduced
for each observation, and we denote them \(w_{ijk\ell m}\). These
variables are defined as follows: \[y_{ijk\ell m}=
  \begin{dcases}
      0 & \text{ then }  w_{ijk\ell m} < 0 \\ 
      1  & \text{ then }  w_{ijk\ell m} \geq 0
  \end{dcases}
\]

Without loss of generality, the variable \(w_{ijk\ell m}\) is assumed to
be distributed as a normal with a variance of one and the mean of
\(\mu_{ijk\ell}\). Thus, the parameters of interest at this level are
\(w_{ijk\ell m}\) and \(\mu_{ijk\ell}\). It is easy to compute the
posterior of \(\boldsymbol{w}\) given the vector of \(\boldsymbol{\mu}\)
and, conversely, it is easy to compute the posterior of
\(\boldsymbol{\mu}\) given the \(\boldsymbol{w}\)'s. Hence, the
marginals of each may be found by Markov chain Monte Carlo (MCMC)
sampling. The conditional posterior distribution of \(\boldsymbol{w}\)
is a truncated normal: \[w_{ijk\ell m}|y_{ijk\ell m} \sim 
  \begin{dcases} 
    \text{N}_-(\mu_{ijk\ell},1) & \text{ if }  y_{ijk\ell m}=0 \\
    \text{N}_+(\mu_{ijk\ell},1) & \text{ if }  y_{ijk\ell m}=1, 
  \end{dcases}
\] where \(N_+\) and \(N_-\) denote normals distributions truncated at
zero from below and above, respectively.

Posterior distributions for remaining parameters may also be sampled
with MCMC. In all cases, conditional posterior distributions of
parameters may be derived directly from the proportional form of Bayes
rule (Jackman, 2009; Rouder \& Lu, 2005). Priors were chosen to be
conjugate, and consequently, posterior distributions may be sampled from
known distributions in Gibbs steps.

Critical parameters are individual estimates of contrast and
assimilation, and the population-level correlation of these abilities.
Individual estimates are parameters \(\alpha_i\) and \(\beta_i\).
Posterior means and posterior standard deviations of these parameters
are shown as points and ellipses, respectively, in Figure 4A in the main
paper. The most critical parameter is the population correlation. On
each iteration of the MCMC chain, we calculated
\(\rho=\Sigma_{1,2}/\sqrt{\Sigma_{1,1}\times \Sigma_{2,2}}\). The prior
and posterior distribution of \(\rho\) is shown in Figure 4B. A full
description of the results may be found in the main paper.

\newpage

\section{References}\label{references}

\setlength{\parindent}{-0.5in} \setlength{\leftskip}{0.5in}

\hypertarget{refs}{}
\hypertarget{ref-Albert:Chib:1995}{}
Albert, J. H., \& Chib, S. (1995). Bayesian residual analysis for binary
response regression models. \emph{Biometrika}, \emph{82}, 747--759.

\hypertarget{ref-Jackman:2009}{}
Jackman, S. (2009). \emph{Bayesian analysis for the social sciences}.
Chichester, United Kingdom: John Wiley \& Sons.

\hypertarget{ref-Rouder:2016}{}
Rouder, J. N. (2016). The what, why, and how of born-open data.
\emph{Behavioral Research Methods}, \emph{48}, 1062--1069. Retrieved
from \url{10.3758/s13428-015-0630-z}

\hypertarget{ref-Rouder:Lu:2005}{}
Rouder, J. N., \& Lu, J. (2005). An introduction to Bayesian
hierarchical models with an application in the theory of signal
detection. \emph{Psychonomic Bulletin and Review}, \emph{12}, 573--604.

\hypertarget{ref-Rouder:etal:2007a}{}
Rouder, J. N., Lu, J., Sun, D., Speckman, P. L., Morey, R. D., \&
Naveh-Benjamin, M. (2007). Signal detection models with random
participant and item effects. \emph{Psychometrika}, \emph{72}, 621--642.






\end{document}
